%%%%%%%%%%%%%%%%%%%%%%%%%%%%%%%%%%%%%%%%%
% Short Sectioned Assignment
% LaTeX Template
% Version 1.0 (5/5/12)
%
% This template has been downloaded from:
% http://www.LaTeXTemplates.com
%
% Original author:
% Frits Wenneker (http://www.howtotex.com)
%
% License:
% CC BY-NC-SA 3.0 (http://creativecommons.org/licenses/by-nc-sa/3.0/)
%
%%%%%%%%%%%%%%%%%%%%%%%%%%%%%%%%%%%%%%%%%

%----------------------------------------------------------------------------------------
%	PACKAGES AND OTHER DOCUMENT CONFIGURATIONS
%----------------------------------------------------------------------------------------

\documentclass[paper=a4, fontsize=11pt]{scrartcl} % A4 paper and 11pt font size

\usepackage[T1]{fontenc} % Use 8-bit encoding that has 256 glyphs
%\usepackage{fourier} % Use the Adobe Utopia font for the document - comment this line to return to the LaTeX default
\usepackage[english]{babel} % English language/hyphenation
\usepackage[utf8]{inputenc}  %allows non-English characters
\usepackage{amsmath,amsfonts,amsthm} % Math packages

\usepackage{sectsty} % Allows customizing section commands
%\allsectionsfont{\centering \normalfont\scshape} % Make all sections centered, the default font and small caps
\allsectionsfont{\centering}

\usepackage{fancyhdr} % Custom headers and footers
\pagestyle{fancyplain} % Makes all pages in the document conform to the custom headers and footers
\fancyhead{} % No page header - if you want one, create it in the same way as the footers below
\fancyfoot[L]{} % Empty left footer
\fancyfoot[C]{} % Empty center footer
\fancyfoot[R]{\thepage} % Page numbering for right footer
\renewcommand{\headrulewidth}{0pt} % Remove header underlines
\renewcommand{\footrulewidth}{0pt} % Remove footer underlines
\setlength{\headheight}{13.6pt} % Customize the height of the header



%\numberwithin{equation}{section} % Number equations within sections (i.e. 1.1, 1.2, 2.1, 2.2 instead of 1, 2, 3, 4)
%\numberwithin{figure}{section} % Number figures within sections (i.e. 1.1, 1.2, 2.1, 2.2 instead of 1, 2, 3, 4)
%\numberwithin{table}{section} % Number tables within sections (i.e. 1.1, 1.2, 2.1, 2.2 instead of 1, 2, 3, 4)

%\setlength\parindent{0pt} % Removes all indentation from paragraphs - comment this line for an assignment with lots of text

\usepackage{caption}

\usepackage{graphicx}
\graphicspath{{../images/}}


%shortcuts for typing variance and expectation 
\newcommand{\E}{\mathrm{E}}
\newcommand{\Var}{\mathrm{Var}}

%----------------------------------------------------------------------------------------
%	TITLE SECTION
%----------------------------------------------------------------------------------------

\newcommand{\horrule}[1]{\rule{\linewidth}{#1}} % Create horizontal rule command with 1 argument of height

\title{	
\normalfont \normalsize 
\textsc{UPC} \\ [25pt] % Your university, school and/or department name(s)
\horrule{0.5pt} \\[0.4cm] % Thin top horizontal rule
\huge Lab3: Significance of Network Metrics \\ % The assignment title
\horrule{2pt} \\[0.5cm] % Thick bottom horizontal rule
}

\author{Simon Van den Eynde\\ Martí Renedo Mirambell} % Your name

\date{\normalsize\today} % Today's date or a custom date

\begin{document}


\maketitle % Print the title


%----------------------------------------------------------------------------------------
%	INTRO
%----------------------------------------------------------------------------------------

\section{Introduction}


\section{Results}

\section{Discussion}

\section{Methods}

\subsection{Analytical estimation of the $p$-value}
\subsubsection{Erdös-Rényi Graph}
Given our original graph with $N$ vertices and $E$ edges, the Erdös-Rényi graph has the same size and order but with its edges randomized.  Of the two possible Erdös-Renyi models, we will use $G(N,p)$ where $p$ will be such that the expected number of edges is $E$ (which is $p=\frac{E}{\binom{N}{2}}$). This has the advantage that $X_j$ the random variables that for every possible vertex indicate wether it exists ($X_j$=1) or not ($X_j$=0) are independent, which will be very useful later on. When working with large values of $N$ and $E$ (such as those we study in this lab session), the models $G(N,E)$ and $G(N,p)$ should give very similar random graphs. 
For $X_j$ where $j$ is the index of any vertex, we calculate its expectation and variance: 
$$\E[X_j]=0\cdot(1-p)+1\cdot p=p$$
$$\Var(X_j)=\E[X^2]-(\E[X])^2=p-p^2$$

Given a vertex $v$ of the graph, it can have $N-1$ neighbours (each with probability $p$), which results in $\binom{N-1}{2}$ possible pairs of neighbours. The expected number of pairs. This gives an expected number of pairs of neighbours $p^2\binom{N-1}{2}$. We will need to work with





\end{document}